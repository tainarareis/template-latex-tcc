\chapter[Referencial teórico]{Referencial teórico}\label{ch:referencial}

(a fazer)


\section{Agentes}

Existem diferentes definições para agentes. Ainda assim, estas definições possuem conceitos básicos em comum como sugere \citeauthor{mcarthur2007multi} (\citeyear{mcarthur2007multi}): a noção de agente, seu ambiente e autonomia. Um agente é uma entidade de software ou hardware capaz de reagir de forma autônoma às alterações do ambiente o qual está situado \apud{wooldridge1999intelligent}{mcarthur2007multi}.


Agentes podem ser representadas por seis características ortogonais como sugere a Figura \ref{fig:hexagono}. Trabalhando juntas, tornam o agente mais robusto e suscetível a mudanças. Quanto maior for a área fechada no diagrama, mais "\textit{agent-like}" é aquele componente do sistema \cite{griss2001software}.

\begin{figure}[h!]
    \includegraphics[scale=0.7]{figuras/hexagono_agente}
    \centering
    \caption{Dimensões do agente. Fonte: \cite{griss2001software}. Traduzido.}
    \label{fig:hexagono}
\end{figure}

Tais características são detalhadas a seguir:

\begin{enumerate}
    \item Adaptabilidade: o grau em que o comportamento de um agente pode ser alterado depois de ter sido implantado.
    \item Autonomia: o grau em que um agente é responsável pelo seu próprio processo de controle e pode prosseguir a sua própria meta em grande parte independente de mensagens enviadas de outros agentes.
    \item Cooperação: o grau em que os agentes se comunicam e trabalham em cooperação com outros agentes para formar sistemas multi-agentes que trabalham juntos em alguma tarefa.
    \item Inteligência: o grau em que um agente é capaz de raciocinar sobre suas metas e seus conhecimentos.
    \item Mobilidade: a capacidade de um agente para passar da execução de contexto para outro, seja movendo o código do agente e iniciar o agente de novo, ou serializando código e estado, permitindo que o agente continue a execução em um novo contexto, mantendo o seu estado para continuar o seu trabalho.
    \item Persistência: trata-se do grau em que a infra-estrutura permite que os agentes conservem conhecimento e estado durante longos períodos de tempo, incluindo robustez diante de possíveis falhas em tempo de execução.
\end{enumerate}

\section{Sistemas Multiagentes}

Sistemas multiagentes se enquadram no cenário onde coexistem dois ou mais agentes. Seus objetivos individuais  correspondem às subpartes do que do objetivo geral do sistema \cite{mcarthur2007multi}. Torna-se necessário para um agente representar e raciocinar sobre os outros agentes no ambiente \cite[pág. 887]{van2008handbook}. 


\section{Padrão}

(a fazer)

\section{Catálogo de padrões}

Um catálogo de padrões é um conjunto de padrões relacionados; por vezes vagamente ou informalmente relacionados. No catálogo, os padrões são subdividios em categorias abrangentes podendo incluir referências cruzadas entre os padrões \cite{appleton1997}. O catálogo de padrões pode incluir a apresentação da estrutura e organização do padrão, mas normalmente não apresenta algo além da estrutura e das relações mais  facilmente identificáveis \cite{appleton1997}.

A exemplo de \citeauthor{gamma1995} (\citeyear{gamma1995}), foram organizaos 23 padrões de projeto orientado a objetos em um catálogo. Este catálogo descreve boas soluções de software de uso genérico, ou seja, padrões que independem do domínio de aplicação. Os autores utilizaram dois critérios para classificar os padrões de projeto: (i) escopo e (ii) propósito. Assim pôde-se comunicar e exibir as relações entre os padrões.

Portanto, em resumo, para caracterizar um catálogo de padrões deve-se fornecer:

\begin{itemize}
    \item Categorias de padrões;
    \item Critérios de classificação do padrão (em categorias);
    \item Relacionamentos/Referências cruzadas entre padrões.
\end{itemize}

\section{Componente}



De acordo com o SWEBOK \cite{swebok} um componente de software é uma unidade independente do software. Esta unidade é compreendida em interfaces e dependências bem definidas. Corroborando com esta ideia, \citeauthor{buschmann2007} define componente como parte independente, destacável e executável de um software. Sua responsabilidade é implementar um serviço específico ou um conjunto de serviços de outros componentes ou clientes. 

Um componente provê uma ou mais interfaces que permitem o acesso aos seus serviços. Podem ser comparados a "blocos de construção" para a estruturação de um sistema. Embora um componente seja independente, componentes podem possuir dependências ou ser compostos de outros componentes. Para \citeauthor{buschmann2007} (\citeyear{buschmann2007}), os componentes podem ser representados como módulos, classes ou um conjunto de funções relacionadas. 

\section{Arquitetura de software}

Arquitetura é a estrutura ou organização de componentes 

\begin{citacao}
"A arquitetura de software é uma descrição dos subsistemas e componentes de um sistema de software e as relações entre eles. Subsistemas e componentes são muitas vezes especificados por meio de diferentes pontos de vista para mostrar as propriedades funcionais e não funcionais relevantes de um sistema de software. A arquitetura de um sistema de software é um artefato que resulta de atividades de projeto de software \cite{buschmann2007}."
\end{citacao}

\section{Arquitetura de Sistemas Multiagentes}


(a fazer)

\section{Estruturas organizacionais}

\cite{kolp2006}
