%% Protocolo de Revisão Sistemática

\section{Tema}

O objetivo dessa revisão sistemática é identificar modelos arquiteturais organizacionais para sistemas multiagentes. Tais modelos são vistos como candidatos a padrões arquiteturais para SMA.

\section{Problema}

Não existem estudos que reúnam modelos arquiteturais para SMA suficientes e que tratem o nível organizacional. E, consequentemente não são muito bem conhecidas as características destes modelos.

\subsection{Questões de pesquisa}

Serão investigadas as respostas para as seguintes questões:

\begin{enumerate}
    \item Quais são os modelos arquiteturais de nível organizacional para SMA?
    \item Quais são as características dos modelos arquiteturais identificados?
\end{enumerate}


\subsection{Intervenção}
Serão observados modelos de arquitetura de SMA que se enquadram como estruturas organizacionais.

\subsection{População}
Áreas de aplicação e pesquisa de SMA serão observada durante a intervenção.

\subsection{Controle}
O conjunto de dados iniciais utilizados pelos pesquisadores são artigos que abordam diretamente o objeto de estudo, são eles:

\begin{itemize}
    \item \textit{Multi-agent architectures as organizational structures} \cite{kolp2006};
    \item \textit{Organizational Multi-Agent Architectures for Information Systems} \cite{do2003};
    \item \textit{Organizational multi-agent architectures: a mobile robot example} \cite{kolp2002}.
\end{itemize}


\subsection{Resultados}
Espera-se a obtenção de modelos arquiteturais organizacionais para SMA que sejam potencialmente classificáveis como padrões arquiteturais.

\subsection{Aplicação}
Serão beneficiadas áreas de aplicação de SMA; especificamente em processos de desenvolvimento de software utilizando como arquitetura estruturas organizacionais. 
%Etapas posteriores da condução deste TCC também são beneficiadas.


\subsection{Palavras-chave e sinônimos}

Foram consideradas como palavras-chave as seguintes palavras na língua inglesa:

\begin{enumerate}
    \item Multi-agent* architetur*, multiagent* arquitectur*;
    \item Organization* structur*, organisation* structur*.
\end{enumerate}

\section{Seleção de fontes de pesquisa}
Os seguintes critérios foram adotados para a seleção das fontes de pesquisa:

\begin{enumerate}
    \item Dentre as áreas de domínio das fontes de pesquisa deverão se enquadrar à area da Computação ou Engenharia de Software;
    \item Os artigos deverão estar disponíveis gratuitamente ou as fontes deverão ser conveniadas com a UnB;
    \item Deve disponibilizar os artigos eletrônicamente;
    \item Deve conter artigos em inglês e português.
\end{enumerate}

\subsection{Critérios para seleção de fonte}

\subsection{\textit{Strings} de pesquisa}

A \textit{string} de pesquisa final obtida após X iterações da RS foi a seguinte:

%“Agent Architecture”, “Software Agent Architecture”, “Multiagent Architecture”, “Multiagent Systems Architecture”

\subsection{Lista de fontes de pesquisa}

As fontes de pesquisa adotadas são:
\begin{enumerate}
    \item ACM Digital Library\footnote{http://http://dl.acm.org};
    \item IEEE Computer Science Digital Library\footnote{http://ieeexplore.ieee.org};
    \item Springer Link\footnote{http://link.springer.com}.
\end{enumerate}

\section{Seleção de trabalhos}

\subsection{Critérios de inclusão}

Serão adotados os seguintes critérios para incluir artigos aos resultados:

\begin{enumerate}
    \item O artigo deve ser escrito em inglês ou português;
    \item O artigo deve estar disponível em meio eletrônico;
    \item O artigo deve apresentar no mínimo um modelo arquitetural organizacional para SMA descrevendo suas características de forma explícita; e/ou apresentar relato de experiências com tais modelos; e/ou comparar modelos.
\end{enumerate}

\subsection{Critérios de exclusão}

Serão adotados os seguintes critérios para excluir artigos dos resultados:

\begin{enumerate}
    \item O artigo não aborda SMA;
    \item O artigo não aborda arquiteturas de SMA;
    \item O artigo não aborda o nível organizacional da arquitetura de SMA;
    \item O artigo não pertence ao domínio da computação ou engenharia de software.
\end{enumerate}

\subsection{Procedimento para seleção dos trabalhos}


\section{Coleta de dados}

\section{Indicadores}






Alterações Relativas a Segunda Iteração da Revisão
Sistemática








