%% Protocolo de Revisão Sistemática

\section{Tema}

O objetivo dessa revisão sistemática é identificar modelos arquiteturais organizacionais para sistemas multiagentes. Tais modelos são vistos como candidatos a padrões arquiteturais para SMA.

\section{Problema}

Não existem estudos que reúnam modelos arquiteturais para SMA suficientes e que tratem o nível organizacional. E, consequentemente não são muito bem conhecidas as características destes modelos.

\subsection{Questão}

Serão investigadas as respostas para as seguintes questões:

\begin{enumerate}
    \item Quais são os modelos arquiteturais de nível organizacional para SMA?
    \item Quais são as características dos modelos arquiteturais identificados?
\end{enumerate}

\subsection{Intervenção, população e tals}

\subsection{Palavras-chave e sinônimos}

Foram consideradas como palavras-chave as seguintes palavras na língua inglesa:

\begin{\begin{enumerate}
    \item Multi-agent* architetur*, multiagent* arquitectur*;
    \item Organization* structur*, organisation* structur*;
\end{enumerate}

\section{Seleção de fontes de pesquisa}

\subsection{Critérios para seleção de fonte}

\subsection{Idiomas}

\subsection{Strings de pesquisa}


“Agent Architecture”, “Software Agent Architecture”,
“Multiagent Architecture”, “Multiagent Systems Architecture”

\subsection{Lista inicial de fontes de pesquisa}

\section{Seleção de trabalhos}

\subsection{Critérios de inclusão}

\subsection{Critérios de exclusão}

\subsection{Procedimento para seleção dos trabalhos}

\section{Coleta de dados}

\section{Indicadores}






Alterações Relativas a Segunda Iteração da Revisão
Sistemática








