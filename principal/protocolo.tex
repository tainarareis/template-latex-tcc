%% Protocolo de Revisão Sistemática

\section{Tema}

O objetivo dessa revisão sistemática é identificar padrões arquiteturais organizacionais para sistemas multiagentes.

Emergir, com base na literatura, vários tipos/modelos arquiteturais para SMA. Tais tipos são vistos como candidatos a padrões arquiteturais para SMA.

\section{Problema}

Não existem estudos que reúnam padrões arquiteturais para SMA suficientes que tratem o nível organizacional. E, consequentemente não são muito bem conhecidas as características destes padrões.

\subsection{Questão}

Serão investigadas as respostas para as seguintes questões:

\begin{enumerate}
    \item Quais são os padrões arquiteturais de nível organizacional para sistemas multiagentes?
    \item Quais são as características dos padrões arquiteturais identificados?
\end{enumerate}

\subsection{Palavras-chave e sinônimos}

\section{Seleção de fontes de pesquisa}

\subsection{Critérios para seleção de fonte}

\subsection{Idiomas}

\subsection{Strings de pesquisa}


“Agent Architecture”, “Software Agent Architecture”,
“Multiagent Architecture”, “Multiagent Systems Architecture”

\subsection{Lista inicial de fontes de pesquisa}

\section{Seleção de trabalhos}

\subsection{Critérios de inclusão}

\subsection{Critérios de exclusão}

\subsection{Procedimento para seleção dos trabalhos}

\section{Coleta de dados}

\section{Indicadores}















