\chapter[Introdução]{Introdução}\label{ch:introducao}

\section{Contextualização}

Agentes de software são uma área de rápido crescimento dentro da TI \cite{jennings1996}. São usados para aplicações diversas como comércio eletrônico, design de interface, jogos e gestão de processos industriais e comerciais complexos \cite{jennings1996}.



\section{Justificativa}

benefícios

\section{Questões de pesquisa}

\section{Objetivos}

\subsection{Objetivo geral}

catálogo de padrões
arquiteturais para SMA.

\subsection{Objetivos específicos}

\begin{enumerate}
    \item Identificar modelos arquiteturais de SMA candidatos à padrões arquiteturais;
    \item Selecionar modelos arquiteturais que se caracterizam como padrões arquiteturais; 
    \item Modelar os padrões arquiteturais usando notação apropriada;
    \item Implementar os modelos utilizando intencionalidade ou comportamento;
    %\item Estudar paradigmas de SMA: Intencionais (Modelo BDI, por exemplo) e Comportamentais (Behaviours). 
\end{enumerate}

\section{Organização dos capítulos}
    
Este trabalho de conclusão de curso está organizado em sete capítulos, descritos brevemente a seguir:

\begin{itemize}
    \item \textbf{Capítulo \ref{ch:introducao} - Introdução:} apresenta o contexto, justificativa, questões de pesquisa e objetivos.
    \item \textbf{Capítulo \ref{ch:referencial} - Referencial teórico:} descreve os conceitos de engenharia de software que fundamentam o trabalho.
    \item \textbf{Capítulo \ref{ch:suporte} - Suporte tecnológico:} apresenta as ferramentas que suportarão as atividades de desenvolvimento de software, gerenciamento, documentação, dentre outras.
    \item \textbf{Capítulo \ref{ch:proposta} - Proposta:} detalha os procedimentos de seleção de padrões de projeto e classificação dos mesmos.
    \item \textbf{Capítulo \ref{ch:metodologia} - Metodologia:} estabelece os procedimentos a serem seguidos do início até a conclusão do trabalho;
    \item \textbf{Capítulo \ref{ch:resultados} - Resultados parciais:} apresenta os resultados alcançados durante o TCC1.
    \item \textbf{Capítulo \ref{ch:consideracoes} - Considerações finais:} relata o status do trabalho alcançado até a execução do TCC1 e os resultados esperados para o TCC2.
\end{itemize}





