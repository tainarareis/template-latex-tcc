\chapter[Introdução]{Introdução}\label{ch:introducao}

\section{Contextualização}

Computadores oferecem pouca ajuda para tarefas complexas. Atualmente, é apenas uma entidade passiva à espera de executar instruções a partir do teclado, mouse ou toque. Tais instruções precisam ser específicas e detalhadas para que sejam executadas com eficiência \cite{green1997}.

Por este motivo, agentes de software despertam o interesse de diversos pesquisadores e empresas de software \cite{green1997}. Trata-se de uma área de rápido crescimento dentro da TI \cite{jennings1996}. Para \citeonline{green1997}, agentes de software se fazem necessários em dias atuais visto que tarefas diárias das pessoas baseiam-se em computadores. 

Tais agentes "conhecem" os interesses dos usuários e podem agir de forma autônoma em prol dos mesmos. Ao invés de exercer controle completo, o papel dos usuários se dá de forma cooperativa com agentes de modo a cumprir suas metas \cite{green1997}.

 %Outro fator ressaltado pelo autor é forma dinâmica e não estruturada com que informações são geradas.%

Sistemas multiagentes encontram espaço em uma série de diferentes domínios de aplicação \cite[pág. 246]{livrao}. São usados para aplicações como comércio eletrônico, design de interface, jogos e gestão de processos industriais e comerciais complexos \cite{jennings1996}. \citeonline{green1997} aponta outras aplicações como controle de tráfego aéreo, mineração de dados (\textit{data mining}), recuperação e gestão da informação, educação e assistentes digitais pessoais (\textit{Personal Digital Assistants}, PDAs).



Uma das armadilhas apontadas por \citeonline[pág. 246]{livrao} é quando agentes interagem muito livremente ou de forma desorganizada. A dinâmica de sistemas multiagentes são complexas, e pode se tornar caótica. Para descobrir o que acontecerá em seguida no sistema, muitas vezes é necessário executar o sistema repetidamente. O número de agentes também influencia em termos de complexidade para gerir de forma eficaz o sistema \cite[pág. 246]{livrao}. 

Em vista disso, uma das formas de estruturar o sistema multiagente está em reduzir a complexidade do mesmo. Por conseguinte, a eficiência do sistema aumenta, bem como a modelagem do problema o qual o sistema multiagente busca resolver passa a ser mais precisa \cite[pág. 246]{livrao}.

%%%%citar direito%%%

Segundo \citeonline{avgeriou2005}, descrever, encontrar, e aplicar padrões arquiteturais na prática ainda é largamente \textit{ad-hoc} e não sistemático \cite{avgeriou2005}. Isto é devido a várias questões que ainda não foram resolvidas: há uma lacuna semântica sobre o que os padrões arquiteturais realmente representam, qual é a filosofia por trás deles; Além disso, há muita confusão em relação ao que é a granularidade dos padrões arquitetônicos; finalmente, não existe uma classificação aceite ou catalogação de padrões que podem ser usados por arquitetos.

%%%%%%


%%%%%%
Além disso, a escolha da arquitetura de um sistema deve incluir as características da arquitetura até as propriedades do problema; tendo descrições uniformes das arquiteturas disponíveis pode simplificar esta tarefa \cite{shaw1996}.

%%%%%

\section{Justificativa}

benefícios

\section{Questões de pesquisa}

\section{Objetivos}

\subsection{Objetivo geral}

catálogo de padrões
arquiteturais para SMA.

\subsection{Objetivos específicos}

\begin{enumerate}
    \item Identificar modelos arquiteturais de SMA candidatos à padrões arquiteturais;
    \item Selecionar modelos arquiteturais que se caracterizam como padrões arquiteturais; 
    \item Modelar os padrões arquiteturais usando notação apropriada;
    \item Implementar os modelos utilizando intencionalidade ou comportamento;
    %\item Estudar paradigmas de SMA: Intencionais (Modelo BDI, por exemplo) e Comportamentais (Behaviours). 
\end{enumerate}

\section{Organização dos capítulos}
    
Este trabalho de conclusão de curso está organizado em sete capítulos, descritos brevemente a seguir:

\begin{itemize}
    \item \textbf{Capítulo \ref{ch:introducao} - Introdução:} apresenta o contexto, justificativa, questões de pesquisa e objetivos.
    \item \textbf{Capítulo \ref{ch:referencial} - Referencial teórico:} descreve os conceitos de engenharia de software que fundamentam o trabalho.
    \item \textbf{Capítulo \ref{ch:suporte} - Suporte tecnológico:} apresenta as ferramentas que suportarão as atividades de desenvolvimento de software, gerenciamento, documentação, dentre outras.
    \item \textbf{Capítulo \ref{ch:proposta} - Proposta:} detalha os procedimentos de seleção de padrões de projeto e classificação dos mesmos.
    \item \textbf{Capítulo \ref{ch:metodologia} - Metodologia:} estabelece os procedimentos a serem seguidos do início até a conclusão do trabalho;
    \item \textbf{Capítulo \ref{ch:resultados} - Resultados parciais:} apresenta os resultados alcançados durante o TCC1.
    \item \textbf{Capítulo \ref{ch:consideracoes} - Considerações finais:} relata o status do trabalho alcançado até a execução do TCC1 e os resultados esperados para o TCC2.
\end{itemize}





