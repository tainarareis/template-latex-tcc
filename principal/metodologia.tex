\chapter[Metodologia]{Metodologia}

%\section{Considerações iniciais}



\section{Metodologias de pesquisa}

O conhecimento científico diverge dos demais tipos de conhecimento devido à necessidade de adotar fundamentação e metodologias a serem seguidas. Além disso, baseia-se em \textit{“informações classificadas, submetidas à verificação, que oferecem explicações plausíveis a respeito do objeto ou evento em questão”} \cite[pág. 22]{prodanov2013}. Ou seja, é imprescindível determinar o método científico que possibilitou atingir esse conhecimento \cite[pág. 24]{prodanov2013}. 

A pesquisa científica, por sua vez, tem por finalidade descobrir respostas para questões mediante a aplicação do método científico. A pesquisa científica é um \textit{“processo formal e sistemático de desenvolvimento do método científico. O objetivo fundamental da pesquisa é descobrir respostas para problemas mediante o emprego de procedimentos científicos.”}  \cite[pág. 26]{gil2008}.

Segundo \citeauthor{gil2002} (\citeyear{gil2002}, pág. 41), quanto aos objetivos, é possível classificar as pesquisas em
três categorias: (i) exploratória, (ii) descritiva e (iii) explicativa. A Tabela \ref{tab:classificacao_pesquisa} apresenta os objetivos das mesmas.

% ######## init table ########
\begin{table}[h]
 \centering
 \caption{Classificação da pesquisa científica quanto aos objetivos.}
 \label{tab:classificacao_pesquisa}
% distancia entre a linha e o texto
 {\renewcommand\arraystretch{1.25}
 \begin{tabular}{ l l l l }
  \cline{1-1}\cline{2-2}\cline{3-3}\cline{4-4}  
    \multicolumn{1}{|p{1.883cm}|}{\begin{center} 
\end{center}} &
    \multicolumn{1}{p{3.517cm}|}{\begin{center}\textbf{Pesquisa Exploratória}
\end{center}} &
    \multicolumn{1}{p{3.600cm}|}{\begin{center}\textbf{Pesquisa Descritiva}
\end{center}} &
    \multicolumn{1}{p{3.583cm}|}{\begin{center}\textbf{Pesquisa Explicativa}
\end{center}}
  \\  
  \cline{1-1}\cline{2-2}\cline{3-3}\cline{4-4}  
    \multicolumn{1}{|p{1.883cm}|}{\begin{center}\textbf{Objetivo}
\end{center}} &
    \multicolumn{1}{p{3.517cm}|}{\begin{center}Propiciar maior familiaridade com o problema, com o propósito de torná-lo mais explícito ou a construir hipóteses \cite[pág. 41]{gil2002}.
\end{center}} &
    \multicolumn{1}{p{3.600cm}|}{\begin{center}Registrar e descrever os fatos observados sem interferência do pesquisador \cite[pág. 52]{prodanov2013}.
\end{center}} &
    \multicolumn{1}{p{3.583cm}|}{\begin{center}\textit{"Identificar os fatores que determinam ou que contribuem para a ocorrência dos fenômenos”} \cite[pág. 42]{gil2002}.
\end{center}}
  \\  
  \hline

 \end{tabular} }
\end{table}


\section{Classificação da pesquisa}

%Quanto à natureza classifica-se como aplicada pois pretende gerar conhecimentos para aplicação prática, dirigidos à solução de problemas específicos \cite[pág. 35]{silveira}. 

O tipo de pesquisa adotado neste TCC  quanto aos objetivos da pesquisa é o exploratório. Quanto aos procedimentos, inicialmente é realizada pesquisa bibliográfica para permitir que o pesquisador conheça os estudos já realizados sobre o assunto \cite[pág. 31]{fonseca}. 






\section{Metodologia de condução do TCC}

\section{Revisão Sistemática}

\section{Metodologia de desenvolvimento}

\section{Metodologia de análise de resultados}







