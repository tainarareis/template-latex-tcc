\chapter[Metodologia]{Metodologia}

%\section{Considerações iniciais}


\section{Metodologias de pesquisa}

O conhecimento científico diverge dos demais tipos de conhecimento devido à necessidade de adotar fundamentação e metodologias a serem seguidas. Além disso, baseia-se em \textit{“informações classificadas, submetidas à verificação, que oferecem explicações plausíveis a respeito do objeto ou evento em questão”} \cite[pág. 22]{prodanov2013}. Ou seja, é imprescindível determinar o método científico que possibilitou atingir esse conhecimento \cite[pág. 24]{prodanov2013}. 

A pesquisa científica, por sua vez, tem por finalidade descobrir respostas para questões mediante a aplicação de um \textit{“(...) processo formal e sistemático de desenvolvimento do método científico. O objetivo fundamental da pesquisa é descobrir respostas para problemas mediante o emprego de procedimentos científicos.”}  \cite[pág. 26]{gil2008}.

Quanto aos objetivos, é possível classificar as pesquisas em
três categorias \cite[pág. 41]{gil2002}: (i) exploratória, (ii) descritiva e (iii) explicativa. O Quadro \ref{tab:classificacao_pesquisa} apresenta os objetivos das mesmas.

% ######## init table ########
\begin{table}[h]
 \centering
 \caption{Classificação da pesquisa científica quanto aos objetivos.}
 \label{tab:classificacao_pesquisa}
% distancia entre a linha e o texto
 {\renewcommand\arraystretch{0.25}
 \begin{tabular}{ l l l l }
  \hline
    \multicolumn{1}{p{1.883cm}}{\begin{center} 
\end{center}} &
    \multicolumn{1}{p{3.517cm}}{\begin{center}\textbf{Pesquisa Exploratória}
\end{center}} &
    \multicolumn{1}{p{3.600cm}}{\begin{center}\textbf{Pesquisa Descritiva}
\end{center}} &
    \multicolumn{1}{p{3.583cm}}{\begin{center}\textbf{Pesquisa Explicativa}
\end{center}}
  \\  
  \cline{1-1}\cline{2-2}\cline{3-3}\cline{4-4}  
    \multicolumn{1}{p{1.883cm}}{\begin{center}\textbf{Objetivo}
\end{center}} &
    \multicolumn{1}{p{3.517cm}}{\begin{center}Propiciar maior familiaridade com o problema, com o propósito de torná-lo mais explícito ou a construir hipóteses \cite[pág. 41]{gil2002}.
\end{center}} &
    \multicolumn{1}{p{3.600cm}}{\begin{center}Registrar e descrever os fatos observados sem interferência do pesquisador \cite[pág. 52]{prodanov2013}.
\end{center}} &
    \multicolumn{1}{p{3.583cm}}{\begin{center}\textit{"Identificar os fatores que determinam ou que contribuem para a ocorrência dos fenômenos”} \cite[pág. 42]{gil2002}.
\end{center}}
  \\  
  \hline

 \end{tabular} }
\end{table}

Com relação à abordagem, a pesquisa pode ser classificada em: (i) pesquisa quantitativa e (ii) pesquisa qualitativa. A pesquisa quantitativa considera que tudo pode ser quantificável, opiniões e informações podem ser traduzidas em números para que sejam classificadas e analisadas. Este tipo de pesquisa exige o uso de recursos e de técnicas estatísticas \cite{prodanov2013}. A pesquisa qualitativa investiga a relação dinâmica entre o mundo real e o sujeito, isto é, um vínculo inerente entre o mundo objetivo e a subjetividade do objeto de estudo que não pode ser traduzido em números. 

\begin{citacao}
"A utilização desse tipo de abordagem difere da abordagem quantitativa pelo fato de não utilizar dados estatísticos como o centro do processo de análise de um problema, não tendo, portanto, a prioridade de numerar ou medir unidades." \cite{prodanov2013} 
\end{citacao}

\section{Classificação da pesquisa}

%Quanto à natureza classifica-se como aplicada pois pretende gerar conhecimentos para aplicação prática, dirigidos à solução de problemas específicos \cite[pág. 35]{silveira}. 

O tipo de pesquisa adotado neste TCC  quanto aos objetivos da pesquisa é o exploratório. Quanto aos procedimentos, inicialmente é realizada pesquisa bibliográfica para permitir que o pesquisador conheça os estudos já realizados sobre o assunto \cite[pág. 31]{fonseca}. 






\section{Metodologia de condução do TCC}

\section{Revisão Sistemática}

\section{Metodologia de desenvolvimento}

O Scrum é um \textit{framework} no qual pode-se resolver problemas complexos de forma produtiva com o principal objetivo de gerar produtos de maior valor possível \cite{schwaber2016}. Isto é possível empregando-se uma abordagem iterativa e incremental que promove controle de riscos e uma melhor previsão dos mesmos. O Scrum fundamenta-se no empirismo: afirma que o conhecimento é desenvolvido pela experiência. São três os pilares que sustentam a implementação do processo empírico:

\par (i) transparência: aspectos significantes do processo devem ser visíveis para os responsáveis pelos resultados;
\par (ii) inspeção:
\par (iii) adaptação:

O objetivo da Scrum é entregar software de qualidade, tanto quanto possível, em intervalos curtos e fixos de tempo intitulados \textit{Sprints} \cite{beedle1999}. Cada fase do ciclo de desenvolvimento de software - consideradas por \citeauthor{beedle1999} (\citeyear{beedle1999}) como requisitos, análise, projeto, evolução e entrega) - é mapeada para uma \textit{Sprint} ou conjunto de \textit{Sprints} \cite{beedle1999}. Além da \textit{Sprint} existem outros eventos prescritos no Scrum: \textit{Sprint Planning, Daily Scrum, Sprint Review, Sprint Retrospective}. Estes eventos são usados para criar regularidade de encontros do time bem como minimizar a necessidade de reuniões não definidas no Scrum. Cada um desses eventos tem uma duração máxima.



Cada \textit{Sprint} opera com um certo número de itens de trabalho no qual constituem no \textit{Backlog}. Os papéis responsáveis pelo processo é entitulado \textit{Scrum Team}. Este é constituído por: \textit{Product Owner}, \textit{Development Team}, e um \textit{Scrum Master}.


\section{Metodologia de análise de resultados}







