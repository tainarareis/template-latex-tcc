\chapter[Suporte tecnológico]{Suporte tecnológico}\label{ch:suporte}

Este capítulo tem o objetivo de apresentar 

Este capítulo está dividido em duas seções: a Seção x apresenta ..., e a Seção Y apresenta 

\section{Sistemas Multiagentes}


%i$* (lê-se i estrela, significando “intencionalidade distribuída”), framework para modelagem organizacional orientada à meta, ou também conhecida como modelagem intencional.
% http://istar.rwth-aachen.de/tikiview_articles.php
% http://istar.rwth-aachen.de/tikiindex.php?page=i%2A+Tools&structure=i%2A+Wiki+Home

AUML (lê-se Agent UML), uma extensão da UML,
mais específica para modelar Agentes de Software.
o http://waitaki.otago.ac.nz/~michael/auml/

\subsection{Jade}

O JADE, ou \textit{Java Agent Development framework}, é um \textit{framework} voltado ao desenvolvimento de diversas aplicações de agentes a partir da linguagem Java. O JADE seguem o padrão FIPA (Foundation For Intelligent, Physical Agents). O JADE é uma ferramenta desenvolvida pela Telecom Itália em parceria com a Universidade de Parma, onde atualmente é um projeto open source com licença LGPL (Lesser General Public Licence).



\section{Engenharia de software}


As ferramentas a seguir foram utilizadas para auxiliar no gerenciamento e na im-
plementação, bem como na documentação associada. Buscou-se aplicar os conhecimentos
adquiridos no decorrer do curso de engenharia de software para apoiar o processo, tanto
em tecnologia, quanto em práticas de desenvolvimento de software.

\subsection{Gerência de projeto}
As seguintes ferramentas foram selecionadas para dar suporte ao gerenciamento de todo o projeto:

\subsubsection{Bonita BPM}

O Bonita BPM é uma ferramenta \textit{open-source} para gerenciamento de processos de negócios. Ela permite modelar, configurar e executar fluxos de trabalho de negócios utilizando a notação BPMN (\textit{Business Process Management Notation}). No contexto desta monografia, será utilizada para modelar processos relacionados à metodologia (Capítulo \ref{ch:metodologia}). A versão utilizada é a 7.3.

\subsubsection{Trello}



\subsection{Desenvolvimento de software}

As seguintes ferramentas foram selecionadas para apoiar o desenvolvimento:

\subsubsection{Debian}

O Debian é um sistema operacional livre e \textit{open-source} desenvolvido pelo Projeto Debian \cite{debian2016}. Os sistemas desenvolvidos pelo projeto utilizam atualmente o \textit{kernel Linux} ou o \textit{kernel FreeBSD}. O Debian 8 Jessie, licenciado pela \textit{General License Public} (GPL), é utilizado nesta monografia para o desenvolvimento de software.

\subsubsection{Atom}

Atom é um editor de texto livre e \textit{open-source} desenvolvido pelo GitHub. Foi lançado sob a licensa permissiva MIT \textit{License} (\textit{Massachusetts Institute of Technology License}) \cite{atom2016}. O Atom será utilizado para edição de arquivos de código fonte.

\subsubsection{Eclipse IDE}

O Eclipse é uma IDE utilizada principalmente para o desenvolvimento de aplicações Java \cite{eclipse2016}. Está licenciado sob uma licensa \textit{open-source}, a \textit{Eclipse Public License}. %O Eclipse possibilita a customização do ambiente a partir de um sistema de plugins. No contexto deste trabalho%

\subsection{Gerência de configuração de software}

\subsubsection{Git}

Git é uma ferramenta  para controle de versão distribuída sob a licença \textit{GNU General Public License version 2.0}, uma licença \textit{open source}.  Traz benefícios como velocidade, garantia da integridade dos dados e suporte para fluxos de trabalho distribuídos e não-lineares \cite[pág. 31]{chacon2014}. Por estas razões o Git será utilizado para versionamento do código fonte e para a parte escrita do TCC.

\subsubsection{Github}

Github é sistema \textit{online} de hospedagem de código-fonte utilizando Git. Permite a navegalibilidade pelo código-fonte, possui \textit{wikis}, integração com diversas ferramentas e gerenciamento de \textit{issues} \cite{github2016}.

O GitHub será utilizado para gerenciar diferentes linhas de desenvolvimento - as \textit{branchs} - e promover a rastreabilidade dos requisitos - que serão mapeados em \textit{issues} - e seu controle através do método kanban, também presente no GitHub.

\subsection{Pesquisa}
As seguintes ferramentas foram selecionadas para dar suporte à atividades da pesquisa:

\subsubsection{Papeeria}

Papeeria é um editor \textit{online} para LaTeX. O uso é gratuito para a maior parte das funcionalidades. Trata-se de um software proprietário. Para utilizar o Papeeria é necessário somente um \textit{web browser}. Possui o editor de código fonte e o compilador necessários para a documentação. Outro diferencial importante da ferramenta é a colaboração através do GitHub permitindo controle de versão e sincronização em tempo real \cite{papeeria2016}. Toda a documentação da monografia é realizada utilizando o Papeeria.

\subsubsection{LaTeX}
LaTeX é um sistema tipográfico utilizado baseado na linguagem TeX. É adotado por diversas universidades, incluindo a UnB. Suas funcionalidades facilitam a produção de documentação técnica e científica. LaTeX está disponível como software livre e está licenciado sob a LaTeX \textit{Project Public License} (LPPL) na versão 1.3c \cite{latex2016}. O LaTeX foi escolhido para esta monografia por simplificar a formatação da documentação.

\subsubsection{Parsifal}
A condução da RS será apoiada pela ferramenta Parsifal cujo contexto é voltado para realização de revisões sistemáticas de literatura no contexto da Engenharia de Software \cite{parsifal2016}. Parsifal é \textit{online} e gratuito.


