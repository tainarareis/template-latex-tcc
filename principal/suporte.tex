\chapter[Suporte tecnológico]{Suporte tecnológico}\label{ch:suporte}

\section{Modelagem de agentes e SMA}


%i$* (lê-se i estrela, significando “intencionalidade distribuída”), framework para modelagem organizacional orientada à meta, ou também conhecida como modelagem intencional.
% http://istar.rwth-aachen.de/tikiview_articles.php
% http://istar.rwth-aachen.de/tikiindex.php?page=i%2A+Tools&structure=i%2A+Wiki+Home

AUML (lê-se Agent UML), uma extensão da UML,
mais específica para modelar Agentes de Software.
o http://waitaki.otago.ac.nz/~michael/auml/



\section{Engenharia de software}

\subsection{Gerência de projeto}

Bonita BPMN
trello

\subsection{Desenvolvimento de software}

debian 8

atom
vim?

\subsection{Gerência de configuração de software}

\subsubsection{Git}

Git é uma ferramenta  para controle de versão distribuída sob a licença \textit{GNU General Public License version 2.0}, uma licença \textit{open source}.  Traz benefícios como velocidade, garantia da integridade dos dados e suporte para fluxos de trabalho distribuídos e não-lineares \cite[pág. 31]{chacon2014}. Por estas razões o Git será utilizado para versionamento do código fonte e para a parte escrita do TCC.

\subsubsection{Github}

Github é sistema \textit{online} de hospedagem de código-fonte utilizando Git. Permite a navegalibilidade pelo código-fonte, possui \textit{wikis}, integração com diversas ferramentas e gerenciamento de \textit{issues} \cite{github2016}.

O GitHub será utilizado para gerenciar diferentes linhas de desenvolvimento - as \textit{branchs} - e promover a rastreabilidade dos requisitos - que serão mapeados em \textit{issues} - e seu controle através do método kanban, também presente no GitHub.

\subsection{Pesquisa}

papeeria, latex, zotero

\subsubsection{Parsifal}

A condução da RS será apoiada pela ferramenta Parsifal\footnote{https://parsif.al/} cujo contexto é voltado para realização de revisões sistemáticas de literatura no contexto da Engenharia de Software. Parsifal é online e gratuito.


